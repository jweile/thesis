%% Introduction: Length: 25-30 pages.
%%  * Relevant background, 
%%  * Outline of state of knowledge, 
%%  * emphasize outstanding questions 

\chapter{Introduction}

Given the constantly improving cost and speed of genome sequencing, it is reasonable to expect that within the coming decades personal genomes will be known for a substantial part of the global populace. Unfortunately, our limited ability to interpret the variation found within them stands in stark contrast with this development. Even when limiting ourselves to mutations in coding regions of the genome, the effects of most missense variants are not known. While a number of computational approaches exist to make predictions as the effects of coding variants, they are currently not reliable enough for clinical use. Laboratory assays by comparison produce more trustworthy results, but until recently did not scale to the space of all possible mutations. The development of Deep mutational scanning~\cite{fowler_high-resolution_2010} has now made this endeavour possible. In the following sections, each of these issues will be discussed in detail.

\section{The Genotype-Phenotype Problem}

Linking genotype to phenotype is a very difficult problem. The part of the human genome we understand best are protein-coding genes, yet they only constitute a minuscule fraction the whole. Impacts of mutations in other functional elements such as introns, untranslated regions of genes, or regulatory sequences, are more difficult to assay, not to mention the vast stretches of intragenic space. While one might expect the latter to not bear functional significance a priori, its importance is nonetheless highlighted by the the fact that a large number of loci identified as correlated with diseases in genome-wide association studies (GWAS) are found within these regions\todo{Find citation}.
But even for protein-coding sequences the problem is far from simple. Alleles that behave according to the Mendelian model are the exception. Most phenotypes are complex, i.e. emerge through the interplay of many different genetic or environmental factors. Conversely, many genes are also pleiotropic, i.e. they are involved more than one mechanism. Thus, a mutation found in one person may not have the same effect as in another---a phenomenon called incomplete penetrance. Similarly, two different mutations within the same coding sequence need not have the same effect either. Depending on how the translated protein is affected (catastrophic folding failure, alteration of a molecular interaction interface or active site, or a subtle change on an unused surface) the effects may differ in severity or in rare cases even in the emergence of new behaviours.

Given the much greater difficulty of interpreting non-coding regions, clinical applications have so far largely concentrated on protein-coding genes. Sequencing panels for known disease-associated genes and even whole-exome sequencing (WES) are widely commercially available. A number of different standards for classifying mutations with respect to their potential health impacts have been proposed. Most prominently, the American College of Medical Genetics and Genomics (ACMG) standard~\todo{find citation}. It defines categories stretching from ``pathogenic'' via ``variant of uncertain significance'' (VUS) to ``benign''. Even though the mutational landscape for a handful of genes, such as \gene{BRCA1} are explored better than others due to their high monetization potential~\cite{cheon_variants_2014}, the vast majority of clinical variants are currently classified as VUS. For example, over 98\% of missense variants for a gene panel assessing germline cancer risk variants~\cite{maxwell_evaluation_2016} have been discarded as VUS. Not only can these uncertainties unduly burden patients with unnecessary anxiety, they also call into question the value of sequencing in the clinic if the majority of findings are not actionable. With increasing use of WES, this problem is only going to get worse. According to  the 1000 Genomes Project data, every person carries 100-400 missense variants that are so rare that they have likely never been seen before in the clinic~\cite{the_1000_genomes_project_consortium_global_2015}. In the absence of previous observations they would automatically be added to long list of VUS.

\section{\textit{In silico} approaches to variant function assessment}
\label{insilicoIntro}

A number of algorithms exist that offer predictions as to the deleteriousness of mutations, the most prominent ones being PolyPhen-2~\cite{adzhubei_method_2010}, SIFT~\cite{SIFT} and PROVEAN~\cite{choi_fast_2012}. PolyPhen-2 uses a machine learning method using evolutionary conservation and protein structural features. It uses a set of previously reported pathogenic alleles as a positive training set and differences between human genes and their mammalian homologues as a negative training set. SIFT (Sorting Intolerant From Tolerant) by contrast only uses evolutionary conservation. The tool uses multiple sequence alignments to calculate position-specific score matrices for each gene which are then normalized and transformed into probability values. PROVEAN (PROtein Variation Effect ANalyzer) similarly only takes into account sequence alignments. However, rather than just computing a position-specific score, PROVEAN calculates the difference in alignment quality between using the wildtype or variant sequence against clusters of homologous sequences. The average distance is then interpreted as indicative of the deleteriousness of the variant. 

While the three tools succeed in making good predictions, their reliability is unfortunately still not high enough to serve as a basis of clinical decision making. Song Sun and other members of the Roth Lab recently performed an independent comparison of these tools on a set of well established disease-causing variants as well as rare polymorphisms with no known disease association~\cite{sun_extended_2016}. A high precision (the fraction of correct classifications out of all positive classifications) can be considered especially important when considering taking clinical action based on a prediction. When compared at a minimum precision level of 90\%, PolyPhen-2 and PROVEAN only reach a sensitivity of 19\% and 21\% , respectively (where sensitivity is defined as the fraction of correct classifications out of all real existing disease variants). SIFT was not even capable of achieving 90\% precision at any score threshold.

\section{Laboratory approaches to variant function assessment}

An alternative to computational prediction for variant assessment is the use of laboratory assays. Many different types of assays exist that can yield potential insight into the effects of missense variants on protein function. However many of them, such as enzymatic activity assays need to be performed one by one and are not easily scalable. Two particularly useful assays in this respect are Yeast-2-Hybrid and functional complementation.

Yeast-2-Hybrid (Y2H)~\cite{FieldsSongY2H} is a binary protein interaction assay performed within the yeast \species{Saccharomyces cerevisiae}. It is based on the reconstitution of two fragments of the transcription factor Gal4. The Gal4 protein comprises two domains: A DNA-binding (DB) domain and an activating domain (AD) both are required for it to successfully associate with its cognate promoter region and induce expression of a reporter gene downstream of the promoter. When two proteins X and Y are fused to the DB and AD domain respectively, an prospective interaction between X and Y leads to the reconstitution of the transcription factor and subsequently to reporter expression. In most cases, the reporter is an auxotrophy marker, such as \gene{HIS3}, thus linking the ability of the two proteins to interact with each other to the ability of the yeast strain to grow on selective media. When comparing different variants of the same protein interacting with the same partner, reporter expression has even been shown to be proportional to binding affinity~\cite{FieldsY2HAffinity}. This allows for quantitative interpretation of Y2H results under these specific circumstances. 

Y2H does however suffer from a number of drawbacks. Due to the the transcription factor needing to physically associate with DNA, any protein to be examined needs to be able to enter the nucleus. While this issue is alleviated by the fusion with a nuclear localization sequence, it does not work for every protein~\cite{Y2Hnls}. A particular problem are membrane proteins which generally cannot enter the nucleus at all. A variant of Y2H, MYTH exists for these proteins~\cite{MYTH}. It has been estimated that Y2H has an overall assay sensitivity of 20\%. That is, only one in five real existing protein interactions can be detected by Y2H~\cite{Rual2012?}. These levels are comparable most other binary interaction assays, such as LUMIER~\cite{lumier} or MAPPIT~\cite{mappit}.

When considering Y2H as an assay for variant function assessment it is important to consider that it does not measure all aspects of a protein's functionality, but rather only its ability to physically associate with a given interaction partner. Thus only variants that result either in major failures in protein folding or in changes to the binding binding interface could be detected. However, in a recent examination of the Y2H performance of common disease associated variants, we found that approximately two out of three disease variants manifest in such a way~\cite{Sahni2015}. 

%Discuss Edgotyping?

Nonetheless, an assay that can measure the overall functionality of a protein within the cell would be preferable. Functional complementation~\cite{complementation} in yeast offers such an option. It based on the premise that some human genes can be used to rescue the deletion of their orthologues in yeast. That is, a fitness defect resulting from the inactivation of the yeast gene is alleviated by the artificial expression of the human gene. Therefore, any relative changes in fitness upon expressing a variant of the human gene can be interpreted as the variant's effect on the protein's overall ability to function. Song Sun and other members of the Roth Lab have recently examined the applicability of functional complementation in yeast to the assessment of disease variants~\cite{sun_extended_2016}. They have found an astonishing predictive capacity despite yeast and humans being diverged by a 1.5 billion year. Yeast complementation outperformed \textit{in silico} methods like PolyPhen-2 and PROVEAN in terms of disease variant prediction by a wide margin. At the 90\% specificity threshold discussed in section~\ref{insilicoIntro}, the complementation assay achieved a sensitivity of over 60\% (as compared to 19\% and 21\% for the two \textit{in silico} methods, respectively).

The only major drawback of yeast complementation is that currently only 60 human genes have been found to be amenable to the assay~\cite{sun_extended_2016}. %mention future outlook on human cell complementation and CRISPR? 


\section{Deep Mutational Scanning}

Complementation and Y2H promise to be useful tools in the classification of variants of uncertain significance. Yet applying them to retroactively test variants only once they have been found in the clinic would be a slow process. Instead, a proactive approach could prove to be more useful: Building an atlas of the functional effects of all possible variants before they are even seen in the clinic. This would require a massive parallelization of the assays. Indeed such efforts have previously been described. Fowler and Fields have pioneered a technology called Deep Mutational Scanning (DMS)~\cite{fowler_high-resolution_2010}. The original method used phage display as a method of selection on an exhaustive library of mutant proteins followed by deep sequencing to identify which mutant proteins were enriched in the process. While the original method was developed as an extension to alanine scanning~\cite{alanineScanning} in the pursuit of biochemical insights, it can certainly be adapted to work with other selection methods such as Y2H and complementation. Indeed, \todo{Who?} and colleagues have recently adapted it to the use of complementation as a means of selection.

% \section{Edgotyping}

\section{Background: The Sumoylation Pathway}

An ideal context for developing sequence-function technology can be found in the sumoylation pathway. The pathway is of great biological importance, mechanistically complex and has been well studied previously in terms of structure and mechanism. Sumoylation is a protein modification in which a small ubiquitin-like modifier (SUMO) is covalently attached to target proteins in order to modulate their behavior, especially in terms of localization and physical interactions~\cite{sumoylation}. Sumoylation plays an important role in a large number of cellular processes, explaining the essentiality of the core members of the pathway~\todo{REF}. 
Despite employing a distinct set of proteins compared to the ubiquitination machinery, the sumoylation pathway bears many close similarities. Analogously to ubiquitin, an enzymatic cascade of proteases, E1, E2 and E3s guide SUMO through its maturation, activation, conjugation and ligation phase~\cite{sumoylation}. After expression, SUMO is matured through cleavage of four amino acids from its C-terminus, exposing a diglycine motif. Next, an E1 activation complex forms a thioester bond between the SUMO diglycine and a cysteine residue within the E1 protein under the consumption of ATP. The activated SUMO is then transferred to the E2 conjugase (UBE2I) via transesterification. The loaded E2 can recognize target proteins via an exposed motif of four amino acids containing a central lysine~\cite{Sampson2001}. This lysine enters the E2’s active site where it comes into contact with SUMO’s diglycine, forming a peptide bond between its $\epsilon$-amino group and SUMO’s C-terminus~\cite{BernierVillamor2002}. This process can be made more efficient in the presence of some classes E3 proteins or outright forced by others~\cite{Streich2016}. Like ubiquitin, SUMO can also form chains, a process which relies on E2 homodimerization as well as the non-covalent binding of a second SUMO molecule to the loaded E2~\cite{CapiliLima2007,Alontaga2015}. 


\begin{figure}
	\centering
	\includegraphics[width=\textwidth]{img/sumoylation_steps.pdf}
	\caption{Steps in the sumoylation pathway}
\end{figure}


Given the complexity of the Sumoylation system, especially surrounding the E2 component, an examination of sequence-structure-function relationships becomes a multifaceted problem. Mutations could in principle affect any combination of the multiple interaction interfaces which in turn contribute in complex ways to the overall cellular phenotype.
An alanine scan of the yeast SUMO E2 Ubc9 was previously performed and succeeded in identifying functionally important sites within the protein 20. Similarly, a DMS scan of ubiquitin, was previously completed 21. While both of these projects provided great insight into the biochemistry of ubiquitin-like protein pathways, neither has produced a complete map.
%Sampson2001: The small ubiquitin-like modifier-1 (SUMO-1) consensus sequence mediates Ubc9 binding and is essential for SUMO-1 modification. 
%BernierVillamor2002: Structural basis for E2-mediated SUMO conjugation revealed by a complex between ubiquitin-conjugating enzyme Ubc9 and RanGAP1.
%Streich2016: Capturing a substrate in an activated RING E3/E2–SUMO complex
%CapiliLima2007: Structure and analysis of a complex between SUMO and Ubc9 illustrates features of a conserved E2-Ubl interaction
%Alontaga2015: RWD Domain as an E2 (Ubc9)-Interaction Module

